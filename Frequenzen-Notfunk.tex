

\section{Notfunk}
\lit{\href{http://de.wikipedia.org/wiki/Notfunk}{de.wikipedia.org/wiki/Notfunk}}

In Not- oder Katastrophenfällen kann Amateurfunk zur Unterstützung eingesetzt werden. So können auch in Gebieten, die ansonsten von der Kommunikation abgeschnitten sind, zu Koordiations- und Kommunikationszwecken Verbindungen nach draussen aufgebaut werden. Bei Empfang eines Notrufes sollte dieser an Notdienste weitergeleitet werden; gegebenenfalls nach den Koordinaten fragen!

Notzeichen (\texttt{SOS} bzw. \textit{Mayday}, von einem Relais weitergeleitet: \texttt{DDD SOS} bzw. \textit{Relay Mayday}) und Dringlichkeitszeichen (\texttt{XXX} bzw. \textit{Pan}) dürfen von Amateurfunkern nicht eingesetzt werden.


\subsection{Funkbetrieb}
\begin{enumerate}
 \item Funkamateure sind in ihrer Gesamtheit keine Einsatzorganisation, sondern stellen sich einzeln und organisiert in den Dienst der Öffentlichkeit.
 \item Meldet euch im Notfall auf den Notruffrequenzen QRV und sendet nur wenn nötig; es gilt der Grund­satz Funkstille, bis man angesprochen wird
 \item Hört euer nächstes Relais, Simplexfrequenzen und HF-Frequenzen ab
 \item Keine Q-Codes und keine Abkürzungen verwenden
 \item Versucht, allfällige Emotionen zu beherrschen
 \item Befolgt die Anweisungen einer Leitstation
\end{enumerate}

\subsection{Notfunkfrequenzen}
Die folgenden Frequenzen sind die von der IARU empfohlenen Aktivitätszentren für Notfunk. 

\begin{tabular}{r r l}
\bfseries Band & \bfseries Frequenz & \bfseries Details \\
\toprule \arrayrulecolor{rowsep}
\bfseries 80 m  & 3760 kHz & Region 1 \\ \midrule
\bfseries 40 m  & 7060 kHz & Region 1 \\ \midrule
\bfseries 20 m  & 14 300 kHz &  \\ \midrule
\bfseries 17 m  & 18 160 kHz &  \\ \midrule
\bfseries 15 m  & 21 360 kHz &  \\ \midrule
\bfseries 11\,m & 27\,065 kHz & CB-Notfunk (Kanal 9) \\ \midrule
\bfseries 2 m   & 144 260 kHz & USB \\ \midrule
                & 145 500 kHz & FM, Für mobile Stationen \\ \midrule
                & 145 525 kHz & FM \\ \midrule
                & 145 550 kHz & FM \\ \midrule
\bfseries 70 cm & 433 500 kHz & Internationale Aufruffrequenz \\ \midrule
\end{tabular}


\subsection{Notfallmeldung}

\noindent
\begin{tabular}{>{\bfseries} l l}
Wer & Name und Standort des Melders \\ 
Wo & QTH des Notfalls (Ortschaft, Koordinaten) \\ 
Was & Ereignis, welche Hilfe ist nötig? \\ 
Wie viele & Betroffene Personen \\ 
Welche & Verletzungen, Schäden
\end{tabular}

\subsection{Wichtige Telefonnummern HB}
\begin{tabular}{l  >{\bfseries} l}
Notfallnummer & 112 \\ \midrule
Polizei & 117 \\ \midrule
Feuerwehr & 118 \\ \midrule
Ambulanz & 144 \\ \midrule
Vergiftung & 145 \\ \midrule
Rega & 1414 \\ \midrule
Air-Glacier & 1415 \\ \midrule
\end{tabular}

\subsection{Vorrangregeln}
Notfunkverkehr \textit{vor} Verkehr betreffend Ausfall öffentlicher Kommunikationsmittel \textit{vor} regulärem Amateurfunk.