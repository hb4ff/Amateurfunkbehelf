\chapter{Funkverkehr}
\section{Empfangsbeurteilung}\label{sec:rst}
Die Empfangsbeurteilung, der sogenannte Signalrapport, der zwischen zwei Amateurstationen aus­getauscht wird, erfolgt im System «RST».

\vspace{1em}
\begin{tabular}{lll}
\textbf{R} & Readability & (Lesbarkeit) \\
\textbf{S} & Signal Strength & (Signalstärke)\\
\textbf{T} & Tone Quality & (Tonqualität)
\end{tabular}
\vspace{1em}

Die Skala für die Beurteilung der Lesbarkeit eines Signals reicht von 1 bis 5, diejenige für die Stärke und Tonqualität eines Signals jeweils von 1 bis 9.

\vspace{1em}
\begin{tabular}{llll}
 & R – Lesbarkeit & S – Signalstärke & T – Tonqualität \\ \toprule \arrayrulecolor{rowsep}
1 & Nicht lesbar & Kaum hörbar & Äusserst rauer Wechselstromton \\ \midrule
2 & Zeitweise lesbar & Sehr schwach hörbar & Rauer Wechselstromton \\ \midrule
3 & Schwer lesbar & Schwach hörbar & Wechselstromton, leicht klingend \\ \midrule
4 & Lesbar & Ausreichend hörbar & \begin{minipage}[t]{4cm}{\raggedright Gleichgerichteter Wechsel\-strom\-ton, schlecht gefiltert}\end{minipage} \\ \midrule
5 & Gut lesbar & Mässig hörbar & Musikalisch modulierter Ton \\ \midrule
6 &  & Gut hörbar & Trillerton \\ \midrule
7 &  & Mässig stark hörbar & Unstabiler Gleichstromton \\ \midrule
8 &  & Stark hörbar & \begin{minipage}[t]{4cm}{\raggedright Stabiler Gleichstromton mit etwas Brummodulation}\end{minipage} \\ \midrule
9 &  & Äusserst stark hörbar & Reiner Gleichstromton \\ \midrule
\end{tabular}
\vspace{1em}

Je nach Übertragungsverfahren werden Teile weggelassen, bei Voice-Verbindungen etwa die Tonqualität. Für Beispiele siehe \plink{sec:cwqso}{Muster-QSO in CW} und \plink{sec:pqso}{Muster-QSO Phonie}.

\section{Abkürzungen}
Die im Amateurfunk gebräuchlichen Abkürzungen dienen dazu, den Informationsaustausch effizienter zu gestalten. Die am häufigsten verwendeten Abkürzungen sind:

{

\begin{longtabu} to \linewidth{lll}
\rowfont \bfseries Abkürzung & Bedeutung &  Übersetzung \\
\toprule
\endhead
\arrayrulecolor{rowsep}
ABT & about & ungefähr \\ \midrule
AGN & again & wieder, nochmals \\ \midrule
AM & amplitude modulation & Amplitudenmodulation \\ \midrule
ANI & any & irgendein \\ \midrule
ANT & antenna & Antenne \\ \midrule
B4 & before & vor, vorher \\ \midrule
BCNU & be seeing you & es würde mich freuen, dich  wieder zu treffen \\ \midrule
BK & break & Unterbrechung, unterbreche \\ \midrule
BTR & better & besser \\ \midrule
CFM & confirm & bestätige, ich bestätige \\ \midrule
CONDX & conditions & Bedingungen \\ \midrule
CONGRATS & congratulations & Glückwunsch \\ \midrule
CPI / CPY & copy & aufnehmen \\ \midrule
CQ & come quick & Allgemeiner Anruf \\ \midrule
CS & callsign & Rufzeichen \\ \midrule
CUAGN & see you again & Auf Wiederhören \\ \midrule
CUL & see you later & Bis bald \\ \midrule
CW & continuous wave & Sinuswelle, Morsetelegraphie \\ \midrule
DE & de (franz.) & von \\ \midrule
DR & dear & liebe, lieber \\ \midrule
DWN & down & unten, nach unten \\ \midrule
DX & long distance & grosse Entfernung \\ \midrule
ELE & elements & Elemente \\ \midrule
ES &  & und \\ \midrule
FB & fine business & wunderbar \\ \midrule
FER & for & für \\ \midrule
FM & from & von \\ \midrule
FM & frequency modulation & Frequenzmodulation \\ \midrule
FR & for & für \\ \midrule
FRD & friend & Freund \\ \midrule
FRM & from & von \\ \midrule
GA & good afternoon & Guten Tag (Nachmittag) \\ \midrule
GB & goodbye & Auf Wiederhören \\ \midrule
GD & good day & Guten Tag \\ \midrule
GE & good evening & Guten Abend \\ \midrule
GL & good luck & Viel Glück \\ \midrule
GM & good morning & Guten Morgen \\ \midrule
GN & good night & Gute Nacht \\ \midrule
GND & ground & Masse, Erdung \\ \midrule
GP &  & Groundplane-Antenne \\ \midrule
GUD & good & gut \\ \midrule
HAM & ham & Funkamateur \\ \midrule
HF & high frequency & Kurzwelle, Hochfrequenz \\ \midrule
HI &  & Ich lache \\ \midrule
HPE & hope & Ich hoffe \\ \midrule
HR & here & hier \\ \midrule
HRD & heard & hörte, gehört \\ \midrule
HV & have & haben, Ich habe \\ \midrule
HW & how & wie \\ \midrule
K &  & kommen \\ \midrule
KEY, KY & key & Morsetaste \\ \midrule
LBR &  & Lieber \\ \midrule
LID &  & Schlechter Funker \\ \midrule
LIS & licensed & lizenziert \\ \midrule
LP & long path & Langer Weg \\ \midrule
LSB & lower side band & Unteres Seitenband \\ \midrule
LSN & listen & hören, höre \\ \midrule
LW & long wire & Langer Draht \\ \midrule
MGR & manager & Manager \\ \midrule
MIC, MIKE & microphone & Mikrofon \\ \midrule
MIN & minute & Minute \\ \midrule
MNI & many & viel \\ \midrule
MRI & merry & fröhlich \\ \midrule
MSG & message & Nachricht \\ \midrule
MTR & meter & Meter, Messgerät \\ \midrule
NIL & not in log & Nicht im Log \\ \midrule
NIL &  & Nichts \\ \midrule
NITE & night & Nacht \\ \midrule
NR & near & nahe, In der Nähe von \\ \midrule
NR & number & Nummer \\ \midrule
NW & now & jetzt \\ \midrule
OB & old boy & Alter Junge \\ \midrule
OM & old man & Funkamateur \\ \midrule
OP & operator & Funker \\ \midrule
OT & old timer & Langjähriger Funker \\ \midrule
PEP & peak envelope power & \parbox[t]{4cm}{Spitzenleistung (z. B. Input PEP = 200 W, Output PEP = 100 W)} \\ \midrule
PFX & prefix & Präfix \\ \midrule
PSE & please & bitte \\ \midrule
PSED & pleased & erfreut \\ \midrule
PWR & power & Leistung \\ \midrule
R & roger & verstanden \\ \midrule
RCD, RCVD & received & erhalten, bekommen \\ \midrule
RIG &  & Stationsausrüstung \\ \midrule
RPRT & report & Rapport \\ \midrule
RPT & repeat & wiederhole, ich wiederhole \\ \midrule
RST & readability, strength, tone & Lesbarkeit, Signalstärke, Tonqualität \\ \midrule
RX & receiver & Empfänger \\ \midrule
SKED & schedule & Verabredung \\ \midrule
SN & soon & bald \\ \midrule
SP & short path & Kurzer Weg \\ \midrule
SRI & sorry & Entschuldigung \\ \midrule
SUM & some & etwas \\ \midrule
SWL & short wave listener & Höramateur \\ \midrule
SWR & standing wave ratio & Stehwellenverhältnis \\ \midrule
TEMP & temperature & Temperatur \\ \midrule
TKS, TNX & thanks & Danke \\ \midrule
TRX & transceiver & Sendeempfänger \\ \midrule
TU & thank you & Danke \\ \midrule
TX & transmitter & Sender \\ \midrule
U & you & Sie, du \\ \midrule
UFB & ultra fine business & ganz ausgezeichnet \\ \midrule
UNLIS & unlicensed & Nicht lizenziert, Schwarzfunker \\ \midrule
UP & up & oben, nach oben \\ \midrule
UR & your & dein \\ \midrule
URS & yours & Grüsse \\ \midrule
USB & upper side band & Oberes Seitenband \\ \midrule
UTC & universal time coordinated & Weltzeit \\ \midrule
VIA & via & über, via \\ \midrule
VY & very & sehr \\ \midrule
WID & with & mit \\ \midrule
WKD & worked & arbeitete, gearbeitet \\ \midrule
WX & weather & Wetter \\ \midrule
XCUS & excuse & entschuldige \\ \midrule
XMAS & Christmas & Weihnachten \\ \midrule
XPECT & expect & erwarten \\ \midrule
XYL & ex-young lady & Ehefrau \\ \midrule
YDAY & yesterday & gestern \\ \midrule
YL & young lady & Funkamateurin \\ \midrule
2 & to & zu, nach \\ \midrule
4 & for & für \\ \midrule
33 &  & Viele Grüsse (zwischen (X)YL) \\ \midrule
55 &  & Viel Erfolg \\ \midrule
72 &  & Viele Grüsse (zwischen QRP-Stationen) \\ \midrule
73 &  & Viele Grüsse \\ \midrule
88 & love and kisses & Liebe und Küsse (zwischen OM und YL) \\ \midrule
99 & keep out & Verschwinde
\end{longtabu}
}

\subsection{Zahlen}
«Lange» Zahlen wie 0 und 9 werden oft abgekürzt, wenn klar ist, dass das Zeichen eine Zahl ist (etwa beim Report ist RST immer eine dreistellige Nummer). Abkürzungen für die restlichen Zahlen sind nur sehr selten anzutreffen.

\vspace{1em}
{
\begin{tabular}{cccccccccc}
1 & 2 & 3 & 4 & 5 & 6 & 7 & 8 & 9 & 0 \\
\toprule
a & u & v & 4 & e & 6 & b & d & n & t
\end{tabular}
}
\vspace{1em}

Beispiele: \texttt{rst 599 5nn}, \texttt{pwr 1tt w}

(die RST-Nummer wird oft erst beim zweiten Mal abgekürzt, damit es auch von OMs gelesen werden kann, die das System noch nicht kennen)

\section{Die im Amateurfunk gebräuchlichsten Q-Codes}\label{sec:qcodes}
Q-Codes dienen, wie die Abkürzungen, zum effizienteren Informationsaustausch. Sie werden vor allem im Bereich Dienstverkehr (Aufrechterhaltung der Verbindung, …) verwendet.

Einigen Q-Codes kann ein bejahender oder verneinender Sinn gegeben werden, indem unmittelbar nach der Abkürzung «c» oder «no» übermittelt wird. \textit{(qsk c, qsk no)}.

Die Bedeutung von Q-Codes kann durch Ergänzungen wie Rufzeichen, Ortsnamen, Zeit- und Frequenzangaben etc. erweitert werden. (qrx 1600 14024). Die Stellen, wo solche ergänzende Angaben eingefügt werden, sind in der nachfolgenden Liste mit drei Punkten (…) bezeichnet. 

Die Q-Codes werden zu Fragen, wenn ihnen ein Fragezeichen folgt. Die nachfolgende Liste führt die Bedeutung der Q-Codes sowohl als Frage wie auch als Antwort oder Mitteilung auf. \textit{(qsx?, qsx up 5 to 10)}

Jenen Q-Codes, die mehrere numerierte Bedeutungen haben, ist die entsprechende Nummer unmittelbar nachgestellt. \textit{(qsa1, qrk3)}

{
\setlength{\belowrulesep}{1pt}
\setlength{\aboverulesep}{1pt}
\newcommand{\wrap}[1]{\begin{minipage}[t]{5.5cm}#1\end{minipage}}
\begin{longtabu} to \linewidth {p{.8cm} @{\hspace{6pt}} p{4cm}p{5.5cm}}
\rowfont \bfseries Abk. & Frage & Antwort \\
\toprule
\endhead
\arrayrulecolor{rowsep}
QRA & Wie ist der Name Ihrer Radiostation? & Der Name meiner Radiostation ist … \\  \midrule
QRB & In welcher Entfernung von meiner Station befinden Sie sich ungefähr? & Die Entfernung zwischen unseren Stationen beträgt ungefähr … Seemeilen (oder Kilometer) \\ \midrule
QRG & Wollen Sie mir meine genaue Frequenz (oder die genaue Frequenz von …) mitteilen? & Ihre genaue Frequenz (oder die genaue Frequenz von …) ist … kHz (oder MHz) \\ \midrule
QRH & Schwankt meine Frequenz? & Ihre Frequenz schwankt \\ \midrule
QRI & Wie ist der Ton meiner Aussendung? & Der Ton Ihrer Aussendung ist 1: gut 2: veränderlich 3: schlecht \\ \midrule
QRK & Wie ist die Verständlichkeit meiner Zeichen (oder der Zeichen von …)? & \wrap{Die Verständlichkeit Ihrer Zeichen (oder der Zeichen von …) ist\\ 1: schlecht\\ 2: mangelhaft\\ 3: ausreichend\\ 4: gut\\ 5: ausgezeichnet} \\ \midrule
QRL & Sind Sie beschäftigt? & Ich bin beschäftigt (oder: ich bin mit … beschäftigt). Bitte nicht stören. \\ \midrule
QRM & Werden Sie gestört? & \wrap{Ich werde gestört. oder:\\ Ich werde\\ 1: gar nicht\\ 2: schwach\\ 3: mässig\\ 4: stark\\ 5: sehr stark\\ gestört} \\ \midrule
QRN & Werden Sie durch atmosphärische Störungen beeinträchtigt? & \wrap{Ich werde durch atmosphärische Störungen beeinträchtigt. oder:\\ Ich werde\\ 1: gar nicht \\ 2: schwach\\ 3: mässig\\ 4: stark\\ 5: sehr stark\\ durch atmosphärische Störungen beeinträchtigt.} \\ \midrule
QRO & Soll ich die Sendeleistung erhöhen? & Erhöhen Sie die Sendeleistung. \\ \midrule
QRP & Soll ich die Sendeleistung vermindern? & Vermindern Sie die Sendeleistung. \\ \midrule
QRQ & Soll ich schneller geben? & Geben Sie schneller (… Wörter in der Minute) \\ \midrule
QRS & Soll ich langsamer geben? & Geben Sie langsamer (… Wörter in der Minute) \\ \midrule
QRT & Soll ich die Übermittlung einstellen? & Stellen Sie die Übermittlung ein. \\ \midrule
QRU & Haben Sie etwas für mich? & Ich habe nichts für Sie. \\ \midrule
QRV & Sind Sie bereit? & Ich bin bereit. \\ \midrule
QRX & Wann werden Sie mich wieder rufen? & Ich werde Sie um … Uhr (auf … kHz [oder MHz]) wieder rufen \\ \midrule
QRZ & Von wem werde ich gerufen? & Sie werden von … (auf … kHz [oder MHz]) gerufen. \\ \midrule
QSA & Wie ist die Stärke meiner Zeichen (oder der Zeichen von …)? & \wrap{Die Stärke Ihrer Zeichen (oder der Zeichen von …) ist \\ 1: kaum hörbar \\ 2: schwach\\ 3: ziemlich gut\\ 4: gut\\ 5: sehr gut} \\ \midrule
QSB & Schwankt die Stärke meiner Zeichen? & Die Stärke Ihrer Zeichen schwankt. \\ \midrule
QSD & Ist mein Tasterspiel mangelhaft? & Ihr Tasterspiel ist mangelhaft. \\ \midrule
QSP & Werden Sie an … vermitteln? & Ich werde an … vermitteln. \\ \midrule
QSL & Können Sie mir Empfangsbestätigung geben? & Ich gebe Ihnen Empfangsbestätigung \\ \midrule
QSO & Können Sie mit … unmittelbar (oder durch Vermittlung) verkehren? & Ich kann mit … unmittelbar (oder durch Vermittlung von) verkehren. \\ \midrule
QST & Inoffiziell. Benutzt durch die ARRL (American Radio Relay League) & Allgemeiner Aufruf, der einer Nachricht an alle Radioamateure und ARRL-Mitglieder vorausgeht. \\ \midrule
QSV & Soll ich eine Reihe V auf dieser Frequenz senden? & Senden Sie eine Reihe V auf dieser Frequenz (oder auf … kHz [oder MHz]) \\ \midrule
QSX & Werden Sie Rufzeichen auf der Fre­quenz … hören? & Ich höre Rufzeichen auf der Fre­quenz … \\ \midrule
QSY & Soll ich zum Senden auf eine andere Frequenz übergehen? & Gehen Sie auf eine andere Frequenz über (oder auf … kHz [oder MHz]). \\ \midrule
QSZ & Soll ich jedes Wort oder jede Gruppe mehrmals geben? & Geben Sie jedes Wort oder jede Gruppe zweimal (oder … mal). \\ \midrule
QTC & Wie viele Telegramme haben Sie? & Ich habe … Telegramme für Sie (oder für …) \\ \midrule
QTH & Wie ist Ihr Standort nach Breite und Länge (oder nach jeder anderen Angabe)? & Mein Standort ist … Breite, … Länge (oder nach jeder anderen Angabe). \\ \midrule
QTR & Welches ist die genaue Uhrzeit? & Es ist genau … Uhr 
\end{longtabu}
}

\section{QSOs}
Grundsätzlich gibt es drei Möglichkeiten, ein QSO mit anderen Funkamateuren zu fahren:
\begin{itemize}
 \item Eigener CQ-Ruf
 \item Antwort auf CQ-Ruf einer anderen Station
 \item Bitte um Aufnahme in ein laufendes QSO
\end{itemize}
In QSOs werden wichtige Informationen wie Rufzeichen, Name, qth oder RST bevorzugt mehrmals – in CW je nachdem auch langsamer – gegeben, damit sie mit grösserer Sicherheit korrekt aufgenommen werden können.

Beim Mitschreiben empfiehlt es sich, wichtige Informationen wie Rufzeichen, Name und qth zu unterstreichen, damit man beim Antworten nicht lange danach suchen muss.

Um unsere Zeit in UTC umzurechnen, zieht man während der Winterzeit eine, während der Sommerzeit zwei Stunden ab.

\textbf{Wichtig:} Die folgenden Muster-QSOs sind nicht eigentlich Muster! Ein QSO zwischen Funk­amateuren läuft abgesehen von diesem Grundriss mehr oder weniger nach Lust und Laune ab.



{
\newcommand{\B}[1]{ \hspace*{.4\textwidth }\begin{minipage}{.55\textwidth }#1\end{minipage} }
\newcommand{\T}[1]{\vspace*{1em}

\noindent
\textbf{#1}\\}
\newcommand{\EN}[1]{\textit{#1}}
\definecolor{cgap}{cmyk}{.2,0,0,.4}
\newcommand{\gap}[1]{\textcolor{cgap}{\textit{#1}}}

\newpage
\subsection{Muster-QSO in CW}\label{sec:cwqso}

\textbf{Frequenzsuche und Anfrage, ob diese besetzt ist:}\\
qrl?  (Am besten zweimal fragen)

\T{CQ-Ruf}
cq cq cq de \gap{HB4FF HB4FF HB4FF} pse k\\
\B{HB4FF de HB9DVT HB9DVT  pse k}

\T{1. Durchgang (Begrüssung, Signalrapport, Vorstellung)}
\gap{HB9DVT de HB4FF} =\\
gm dr om es mni tnx fer call =\\
ur rst is \gap{579 579} = \\
my name is \gap{martin martin} es my qth is nr \gap{thun thun} =\\
hw cpi? \gap{HB9DVT de HB4FF} k\\
\B{
HB4FF de HB9DVT =\\
r cpi all =\\
tu fer fb rprt es info =\\
rst is 559 559 =\\
name is thomas thomas qth is nr aarau nr aarau =\\
hw? HB4FF de HB9DVT k
}

\T{2. Durchgang (Stationseinrichtung, Wetter, Sonstiges)}
\gap{HB9DVT} de \gap{HB4FF} =\\
r vy gud cpi dr \gap{thomas} =\\
mni tnx =\\
my rig is \gap{ft-1000mp} pwr abt \gap{100 w} =\\
ant is \gap{dipole} =\\
wx is \gap{overcast} wid temp \gap{16 c} =\\
hw? \gap{HB9DVT de HB4FF} k\\
\B{
r dr martin es mni tks fer info =\\
rig hr is ft-890 pwr is 100 w es ant is r5 vertical =\\
wx is sunny es temp abt 19 c =\\
nw dr martin qru? HB9DVT de HB4FF k
}

\T{3. Durchgang (Bedankung, Verabschiedung)}
\gap{HB9DVT de HB4FF} =\\
ok dr \gap{thomas} mni tnx fer ufb qso =\\
hpe cuagn sn es my qsl sure via buro =\\
best 73 es gb = \gap{HB9DVT de HB4FF} +\\
\B{HB4FF de HB9DVT =\\
tu fer nice qso dr martin =\\
my qsl also ok via buro =\\
gud dx es best wishes =\\
vy 73 cu HB4FF de HB9DVT ++\footnote{+ (ar) und ++ (sk) werden bei Verbindungsende in dieser Reihenfolge gegeben. Zwischen den Buchstaben \textit{a} und \textit{r} bzw. \textit{s} und \textit{k} ist -- wie beim SOS -- keine Pause.}
}


\newpage
\subsection{Muster-QSO Phonie}\label{sec:pqso}
\T{Frequenzsuche und Anfrage, ob diese besetzt ist:}
Ist diese Frequenz belegt?\\
Is this frequency in use?  (Am besten zweimal fragen)

\T{CQ-Ruf}
CQ CQ CQ allgemeiner Anruf von HB4FF Hotel Bravo Vier Foxtrott Foxtrott HB4FF\\
\EN{CQ CQ CQ this is  HB4FF Hotel Bravo Four Foxtrott Foxtrott HB4FF}

\T{1. Durchgang (Begrüssung, Signalrapport, Vorstellung)}
HB9DVT von HB4FF = Guten Tag lieber OM und vielen Dank für den Anruf.\\
Ihr Rapport ist 5 und 9, ein ganz gutes Signal. Mein Name ist ….  und der Standort ist … \\
Zurück zu Ihnen = HB9DVT von HB4FF bitte kommen\\
\EN{HB9DVT this is HB4FF = Very good Morning dear OM and many thanks for the call. \\
Your signal is 5 and 9, fine modulation. My name is … and the qth is …\\
Back to you = HB9DVT this is HB4FF over}

\T{2. Durchgang (Stationseinrichtung, Wetter, Sonstiges)}
HB9DVT von HB4FF = Vielen Dank für die Informationen lieber … (sein Name). Mein Sender ist ein Yaesu FT-1000MP mit etwa 100 W Leistung. Die Antenne ist ein Breitbanddipol.
Das Wetter ist … (sonnig, bewölkt, Regen). Die Temperatur etwa … Grad. Wie ist das angekommen? HB9DVT von HB4FF bitte kommen\\
\EN{HB9DVT this is HB4FF = many thanks for the information dear … (his name). My transceiver is a Yaesu FT-1000MP, the power is 100 Watt. The antenna is a broadband dipol. The weather is … (sunny, cloudy, rain). Temperature is abt … degrades celsius. How do you copy? HB9DVT this is HB4FF over}

\T{3. Durchgang (Bedankung, Verabschiedung)}
HB9DVT von HB4FF = alles klar lieber … (sein Name). Vielen Dank für die Verbindung. 
Ich hoffe, wir treffen uns wieder einmal auf der Frequenz. Meine QSL Karte wird via Büro versendet. 
Alles Gute und bis zum nächsten Mal. Es verabschiedet sich HB4FF Operator … (mein Vorname).\\
\EN{HB9DVT this is HB4FF = all o.k. dear … (sein Name). Many thanks for the qso. 
I hope to meet you again on the frequency. QSL Card is o.k. via Buro.
All the best and hope to meet you again.}\\
Beste 73, tschüss.\\
\EN{73 and good luck. Bye bye}

}

\newpage
\section{Übertragungsverfahren}
Die im Amateurfunk gebräuchlichsten Übertragungsverfahren sind:

\vspace{1em}
\begin{tabular}{llc}
\bfseries Betriebsart & Abkürzung & Bezeichnung \\
\toprule

\bfseries Morsetelegraphie & CW (Continuous wave) & A1A \\ \\

\bfseries Telefonie &  &  \\
\midrule
Amplitudenmodulation & AM & A3E \\
Einseitenbandmodulation & SSB (LSB/USB) & J3E \\
Frequenzmodulation & FM & F3E \\ \\

\bfseries Digitale Übertragungsverfahren &  &  \\
\midrule
Funkfernschreiben & RTTY (Radio teletype) & F1B, J2B \\
Faksimile & FAX & F1C, J3C \\
Packet Radio & PR & F1B \\
Standbildfernsehen & SSTV (Slow scan TV) & J2C \\
Amateurfernsehen & ATV (Amateur TV) & C3F
\end{tabular}
\vspace{1em}

Die Erklärung zu diesen und weiteren Bezeichnungen findet man in den Bakom-Vorschriften unter RR AP 1: Abschnitt II, Sendearten.