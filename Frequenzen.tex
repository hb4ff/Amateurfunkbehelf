
\chapter{Frequenzen}
\section{Amateur-Frequenzbänder}
\subsection{Einteilungsempfehlung der Kurzwellenbänder}
Die Pläne werden für drei Regionen erstellt; Region 1 (Europa), Region 2 (Amerika) und Region 3 (Asien). Die maximale Bandbreite sollte nicht überschritten werden. Die folgende Tabelle entspricht dem Bandplan 2011 von \href{http://www.iaru.org/region-1.html}{iaru.org} für die Region 1.

Automatische (unbeaufsichtigte) Stationen sind nur in den dafür vorgesehenen Frequenzbereichen erlaubt.

Legende: CW = Morse, NB = Schmalband, WB = Alle Betriebsarten, -- =  Senden nicht erlaubt

{
\setlength{\belowrulesep}{1pt}
\setlength{\aboverulesep}{1pt}
\definecolor{nr}{cmyk}{0,1,1,0}
\newcommand{\notruf}[1]{\textcolor{nr}{ #1\,kHz: \textit{Notruffrequenz}}}

\begin{longtabu} to \linewidth {r @{\hspace{4pt}} p{1.6cm} @{} r @{\hspace{4pt}} c @{\hspace{4pt}} p{6.8cm}}
\rowfont \bfseries Band & $f$ (kHz) & Bandbr. & für & Bemerkungen \\
\toprule
\endhead

\bfseries LW & 135.7–137.8 & 200 Hz & NB & Ink. QRSS1 \\ \arrayrulecolor{rowsep} \midrule

\bfseries 160 m & 1810–1838 & 200 Hz & CW & QRP-Zentrum auf 1836 kHz \\ \arrayrulecolor{white} \midrule
 & 1838–1840 & 500 Hz & CW \\ \midrule
 & 1838–1840 & 500 Hz & NB \\ \midrule
 & 1840–1843 & 2700 Hz & WB & + Digital \\ \midrule
 & 1843–2000 & 2700 Hz & WB \\ \arrayrulecolor{rowsep} \midrule

\bfseries 80 m & 3500–3510 & 200 Hz & CW & interkontinentale QSOs bevorzugt \\ \arrayrulecolor{white} \midrule
 & 3510–3560 & 200 Hz & CW & Contest bevorzugt, QRS auf 3555 kHz \\ \midrule
 & 3560–3580 & 200 Hz & CW & QRP auf 3560 kHz \\ \midrule
 & 3580–3600 & 500 Hz & NB & 3590–3600 kHz: Unbeaufsichtigte Stationen\\ \midrule
 & 3600–3650 & 2700 Hz & NB & Telefonie-Contest bevorzugt, SSB auf 3630 kHz\\ \midrule
 & 3650–3700 & 2700 Hz & WB & 3690 kHz: QRP-Anruffrequenz\\ \midrule
 & 3700–3800 & 2700 Hz & WB & Telefonie-Contest­bereich bevorzugt. 3735 kHz: Bilder, \notruf{3760}\\ \midrule
 & 3775–3800 & 2700 Hz & WB & Priorität für interkontinentale Verbindungen \\ \arrayrulecolor{rowsep} \midrule

\bfseries 40 m & 7000–7040 & 200 Hz & CW & 7030 kHz: QRP-Anruffrequenz \\ \arrayrulecolor{white} \midrule
 & 7040–7050 & 500 Hz & NB & Unbeaufsichtigte Stationen ab 7047 kHz \\ \midrule
 & 7050–7060 & 2700 Hz & WB & Unbeaufsichtigt bis 7053 kHz \\ \midrule
 & 7060–7100 & 2700 Hz & WB & SSB-Contest bevorzugt, Aufruf auf 7070 kHz, QRP auf 7090 kHz \\ \midrule
 & 7100–7130 & 2700 Hz & WB & \notruf{7110} \\ \midrule
 & 7130–7200 & 2700 Hz & WB & SSB-Contest bevorzugt, Bilder auf 7165 kHz \\ \midrule
 & 7175–7200 & 2700 Hz & WB & Interkontinental bevorzugt \\ \arrayrulecolor{rowsep} \midrule
\end{longtabu}

\begin{longtabu} to \linewidth {r @{\hspace{4pt}} r @{\hspace{4pt}} r @{\hspace{4pt}} c @{\hspace{4pt}} p{6.4cm}}
\rowfont \bfseries Band & $f$ (kHz) & Bandbr. & für & Bemerkungen \\
\toprule
\endhead
\bfseries 30 m & 10 100–10 140 & 200 Hz & CW & 10 116 kHz: QRP-Anruffrequenz \\ \arrayrulecolor{white} \midrule
 & 10 140–10 150 & 500 Hz & NB \\ \arrayrulecolor{rowsep} \midrule

\bfseries 20 m & 14 000–14 060 & 200 Hz & CW & Contestbereich bevorzugt, QRS auf 14 055 kHz \\ \arrayrulecolor{white} \midrule
 & 14 060–14 070 & 200 Hz & CW & QRP auf 14 060 kHz \\ \midrule
 & 14 070–14 099 & 500 Hz & NB & Unbeaufsichtigt ab 14 089 kHz \\ \midrule
 & 14 099–14 101 &        & -- & Bakenfrequenz exklusive \\ \midrule
 & 14 101–14 125 & 2700 Hz & WB & Unbeaufsichtigt bis 14 112 kHz \\ \midrule
 & 14 125–14 300 & 2700 Hz & WB & SSB-Contest bevorzugt, Voice auf 14 130 kHz. Dxpeditions auf 14 195±5 kHz, Bilder auf 14 230 kHz. QRP-SSB auf 14 285 kHz. \\ \midrule
 & 14 300–14 350 & 2700 Hz & WB & \notruf{14 300} \\ \midrule
 & 14 190–14 200 & 2700 Hz & WB & 14 195 $\pm$ 5 MHz: Dxpeditionen \\ \midrule
 & 14 300–14 350 & 2700 Hz & WB & 14 230 kHz: SSTV/Fax-Anruffrequenz \\ \arrayrulecolor{rowsep} \midrule
 
\bfseries 17 m & 18 068–18 095 & 200 Hz & CW & 18 086 kHz: QRP-Frequenz \\ \arrayrulecolor{white} \midrule
 & 18 095–18 109 & 500 Hz & NB & Unbeaufsichtigt ab 18 105 kHz \\ \midrule
 & 18 109–18 111 &        & -- & Bakenfrequenz – exklusive \\ \midrule
 & 18 111–18 120 & 2700 Hz & WB & Unbeaufsichtigt \\ \midrule
 & 18 120–18 168 & 2700 Hz & WB & QRP-SSB auf 18 130 kHz, Voice auf 18 150 kHz. \notruf{18 160} \\ \arrayrulecolor{rowsep} \midrule
 
\bfseries 15 m & 21 000–21 070 & 200 Hz & CW & QRP auf 21 055 kHz, QRS auf 21 060 kHz \\ \arrayrulecolor{white} \midrule
 & 21 070–21 110 & 500 Hz & NB & Unbeaufsichtigt ab 21 090 kHz \\ \midrule
 & 21 110–21 120 & 500 Hz & WB & Unbeaufsichtigte Stationen erlaubt \\ \midrule
 & 21 120–21 149 & 200 Hz & NB &  \\ \midrule
 & 21 149–21 151 & 200 Hz & -- & Bakenfrequenz – exklusive \\ \midrule
 & 21 151–21 450 & 2700 Hz & WB & Sprache auf 21 180 kHz, QRP-SSB auf 21 285 kHz. Bilder auf 21 340 kHz, \notruf{21 360} \\ \arrayrulecolor{rowsep} \midrule
 
\bfseries 12 m & 24 890–24 915 & 200 Hz & CW & 24 906 kHz: QRP-Frequenz \\ \arrayrulecolor{white} \midrule
 & 24 915–24 929 & 500 Hz & NB & Unbeaufsichtigt ab 24 925 kHz \\ \midrule
 & 24 929–24 931 & 200 Hz & -- & Bakenfrequenz – exklusive \\ \midrule
 & 24 931–24 990 & 2700 Hz & WB & Unbeaufsichtigt bis 24 940 kHz \\ \arrayrulecolor{rowsep} \midrule

\bfseries 10 m & 28 000–28 070 & 200 Hz & CW & QRS auf 28 055 kHz, QRP auf 28 060 kHz \\ \arrayrulecolor{white} \midrule
 & 28 070–28 190 & 500 Hz & NB & Unbeaufsichtigt auf 28 120+30 kHz \\ \midrule
 & 28 190–28 225 &        & -- & Baken; regional bis 28 199 kHz \\ \midrule
 & 28 225–28 320 & 2700 Hz & WB & Baken bis 28 300 kHz, danach unbeaufsichtigt \\ \midrule
 & 28 320–29 100 & 2700 Hz & WB & QRP-SSB: 28 360 kHz. Bilder: 28 680 kHz. \\ \midrule
 & 29 100–29 200 & 6000 Hz & WB & Simplex-FM, 10-kHz-Kanäle (29\,110--29\,290 kHz) \\ \midrule
 & 29 200–29 300 & 6000 Hz & WB & Unbeaufsichtigte Stationen \\ \midrule
 & 29 300–29 510 & 6000 Hz & WB & Satelliten-Downlink exklusive \\ \midrule
 & 29 510–29 520 &         & -- & Schutzkanal \\ \midrule
 & 29 520–29 700 & 6000 Hz & WB & FM: Aufruf auf 29 600 kHz, Relais ab 29 610 kHz \\ \arrayrulecolor{rowsep} \midrule

\end{longtabu}
}

Dass vom 160\,m-Band ausgehend jeweils die zweifache Frequenz wieder in einem Amateurfunkband liegt (80\,m, 40\,m, 20\,m, 10\,m), kommt nicht von ungefähr -- so lagen die Oberwellen, die von älteren Geräten manchmal erzeugt wurden, wieder innerhalb eines Amateurfunkbandes und störte nur andere Amateurfunker.

\subsection{Für Inhaber einer HB-Amateurfunkprüfung}

\begin{minipage}[t]{\textwidth}
\colorbox{white}{
\begin{minipage}[t]{.45\textwidth}
\begin{tabular}{r @{---} l l}
\multicolumn{3}{r}{\bfseries HB3-Lizenz} \\ \toprule \arrayrulecolor{rowsep}
1.810 & 2.000 & MHz \\ \midrule
3.500 & 3.800 & MHz \\ \midrule
21.000 & 21.450 & MHz \\ \midrule
28.000 & 29.600 & MHz \\ \midrule
144.000 & 146.000 & MHz \\ \midrule
430.000 & 440.000 & MHz \\ \midrule
\end{tabular}
\end{minipage}
}
\colorbox{white}{
\begin{minipage}[t]{.45\textwidth}
\begin{tabular}{r @{---} l l}
\multicolumn{3}{r}{\bfseries HB9-Lizenz} \\ \toprule \arrayrulecolor{rowsep}
 135.7 & 137.8 & kHz \\ \midrule
1.810 & 2.000 & MHz \\ \midrule
3.500 & 3.800 & MHz \\ \midrule
7.000 & 7.200 & MHz \\ \midrule
10.100 & 10.150 & MHz \\ \midrule
14.000 & 14.350 & MHz \\ \midrule
18.068 & 18.168 & MHz \\ \midrule
21.000 & 21.450 & MHz \\ \midrule
24.890 & 24.990 & MHz \\ \midrule
28.000 & 29.700 & MHz \\ \midrule
50.000 & 52.000 & MHz \\ \midrule
144 & 146 & MHz \\ \midrule
430 & 440 & MHz \\ \midrule
1.240 & 1.300 & GHz \\ \midrule
2.300 & 2.450 & GHz \\ \midrule
5.650 & 5.850 & GHz \\ \midrule
10.000 & 10.500 & GHz \\ \midrule
\end{tabular}
\end{minipage}
}
\end{minipage}

Auf Bakenfrequenzen ist der Sendebetrieb nicht gestattet, und auf den \link{Notruffrequenzen} sollte nur im Notfall gesendet werden. 

Weitere Informationen (unter anderem zum Verwendungszweck und zur maximalen Leistung von Frequenzbändern) findet man in den Bakom-Vorschriften, Artikel 6. 

\section{Frequenzbereiche}

Über dem EHF-Band beginnt der Infrarotbereich (1…400 THz bzw. 0.74…300 µm) und danach das sichtbare Licht (400…790 THz, 390…790 nm).

{
\newcommand{\freq}[2]{\parbox[t]{6em}{#1\\ \footnotesize #2}}

\begin{longtabu} to \linewidth{l @{\hspace{4pt}} r @{\hspace{6pt}} p{7cm}}
\rowfont \bfseries Frequenz & Wellenlänge & Verwendung \\ \toprule \arrayrulecolor{rowsep}
\endhead
\freq{ELF}{3…30 Hz} & $10^4\dots 10^5$ km & Bereich der Gehirnströme (Alpha-Wellen bis 13\,Hz, Beta-Wellen bis 30\,Hz) \\ \midrule
\freq{SLF}{30…300 Hz} & $10^3…10^4$ km & \parbox[t]{7cm}{(Einweg-)Kommunikation zu U-Booten. Elektromagnetische Frequenzen dieser Wellenlänge können bis etwa 300 m unter Wasser empfangen werden (bei 15 kHz nur noch um 20 m).\footnote{Aus \href{http://de.wikipedia.org/wiki/Längstwelle}{Wikipedia:Längstwelle}} Zum Empfang ziehen U-Boote ein langes Antennenkabel hinter sich her. Aufgrund der tiefen Frequenz (geringe Bandbreite) können nur wenige Informationen übertragen werden. \\ Frequenzen der einzigen 3 Sender: 76 Hz (USA: Wisconsin und Michigan), 82 Hz (RU, Murmansk). Sie verwenden mehrere Kilometer lange Antennen.} \\ \midrule
\freq{ULF}{300…3000 Hz} & 100…1000 km & Wird vor allem auf geringe Distanzen in Bergwerken verwendet.  \\ \midrule
\freq{VLF}{3…30 kHz} & 10…100 km & \parbox[t]{7cm}{Zur Kommunikation mit U-Booten, die sich nahe der Wasseroberfläche befinden. Die Signale können 10 bis 40 m unter der Wasseroberfläche empfangen werden. \\ Bis etwa 10 kHz sind \link{Whistler} hörbar.} \\ \midrule
\freq{LF}{30…300 kHz} & 1…10 km & Zeitzeichensender für Funkuhren (40 bis 80 kHz; DCF77 mit 77.5 kHz, 50 kW), Deutscher Wetterdienst (147.3 kHz, 50 Bd, 85 Hz Shift), Langwellenrundfunk (148.5 bis 283.5 kHz). \\ \midrule
\freq{MF}{300…3000 kHz} & 100…1000 m & See-Notfunk auf 500 kHz, AM-Rundfunk (526.5 bis 1606.5 kHz). Übergang von Boden- zu Raumwellen. \\ \midrule
\freq{HF}{3…30 MHz} & 10…100 m & Militär (Navy). \link{Raumwellen}-Bereich. Die Ausbreitungsbedingungen hängen stark vom Zustand der Ionosphäre ab (siehe Kapitel über \link{Wellenausbreitung}). Mittlere Reichweite zur je idealen Tageszeit: 2500 km (18 MHz), 3500 km (10 MHz), 2000 km (7 MHz), 2500 km (3.5 MHz). \\ \midrule
\freq{VHF}{30…300 MHz} & 1…10 m & FM-Rundfunk, Flugfunk, Fernsehen, Militär. VHF wird von der Atmosphäre praktisch nicht reflektiert. \\ \midrule
\freq{UHF}{300…3000 MHz} & 10…100 cm & Mikrowellen und WLAN um 2.4 GHz (802.11), Handys (GSM, UMTS). Die Wellen werden bereits an Objekten (Hauswänden zum Beispiel) reflektiert. \\ \midrule
\freq{SHF}{3…30 GHz} & 1…10 cm & WLAN auf 5 GHz (802.11a), Satellitenkommunikation, Radar im Luftverkehr\footnote{Radar kann generell auf allen Frequenzen verwendet werden, für unterschiedliche Einsatzzwecke. Niederschlagsradar beispielsweise verwendet Wellenlängen um 3--10\,cm.} und für Lenkwaffen. Frequenzen um 22.2 GHz werden von Wasserdampf stark absorbiert.\footnote{Resonanzabsorption von Wasser: \href{http://de.wikipedia.org/wiki/Resonanzabsorption}{Wikipedia:Resonanzabsorption}} \\ \midrule
\freq{EHF}{30…300 GHz} & 1…10 mm & Datenlinks zwischen zwei Punkten, amerikanisches Active Denial System (unschädliche, aber schmerzhafte Waffe). Frequenzen bei 60 und 118.75 GHz werden von Sauerstoff absorbiert, 183.31 und 325.153 GHz von Wasserdampf \\ \midrule
\end{longtabu}
}



\section{Katastrophenfunk}

Siehe auch: \href{http://de.wikipedia.org/wiki/Notfunk}{de.wikipedia.org/wiki/Notfunk}

\subsection{Funkbetrieb}
\begin{enumerate}
 \item Funkamateure sind in ihrer Gesamtheit keine Einsatzorganisation, sondern stellen sich einzeln und organisiert in den Dienst der Öffentlichkeit.
 \item Meldet euch im Notfall auf den Notruffrequenzen QRV und sendet nur wenn nötig; es gilt der Grund­satz Funkstille, bis man angesprochen wird
 \item Hört euer nächstes Relais, Simplexfrequenzen und HF-Frequenzen ab
 \item Keine Q-Codes und keine Abkürzungen verwenden
 \item Versucht, allfällige Emotionen zu beherrschen
 \item Befolgt die Anweisungen einer Leitstation
\end{enumerate}

\subsection{Weltweite Notfunkfrequenzen}

\begin{tabular}{r r l}
\bfseries Band & \bfseries Frequenz & \bfseries Details \\
\toprule \arrayrulecolor{rowsep}
\bfseries 20 m & 14 300 kHz &  \\ \midrule
\bfseries 17 m & 18 160 kHz &  \\ \midrule
\bfseries 15 m & 21 360 kHz &  \\ \midrule
\bfseries 2 m & 144 260 kHz &  \\ \midrule
 & 145 500 kHz & Für mobile Stationen \\ \midrule
 & 145 525 kHz &  \\ \midrule
 & 145 550 kHz &  \\ \midrule
\bfseries 70 cm & 433 500 kHz & Internationale Aufruffrequenz \\ \midrule
\end{tabular}

\subsection{Lokale Notruffrequenzen}
\begin{tabular}{r r l}
\bfseries Band & \bfseries Frequenz & \bfseries Details \\
\toprule \arrayrulecolor{rowsep}
\bfseries 80 m & 3760 kHz & Region 1 \\ \midrule
\bfseries 40 m & 7060 kHz & Region 1 \\ \midrule
\end{tabular}

\subsection{Notfallmeldung}

\noindent
\begin{tabular}{>{\bfseries} l l}
Wer & Name und Standort des Melders \\ 
Wo & QTH des Notfalls (Ortschaft, Koordinaten) \\ 
Was & Ereignis, welche Hilfe ist nötig? \\ 
Wie viele & Betroffene Personen \\ 
Welche & Verletzungen, Schäden
\end{tabular}

\subsection{Wichtige Telefonnummern HB}
\begin{tabular}{l  >{\bfseries} l}
Notfallnummer & 112 \\ \midrule
Polizei & 117 \\ \midrule
Feuerwehr & 118 \\ \midrule
Ambulanz & 144 \\ \midrule
Vergiftung & 145 \\ \midrule
Rega & 1414 \\ \midrule
Air-Glacier & 1415 \\ \midrule
\end{tabular}

\subsection{Vorrangregeln}
Notfunkverkehr \textit{vor} Verkehr betreffend Ausfall öffentlicher Kommunikationsmittel \textit{vor} regulärem Amateurfunk.



\section{Zeichensender}
\subsection{Baken-Frequenzen}
\paragraph{NCDXF-Bakenfrequenzen} Von der Nothern California DX Foundation wurden auf allen Kontinenten insgesamt 18 Baken verteilt, die auf den folgenden fünf Frequenzbändern regelmässig ihr Rufzeichen senden:

\vspace{1em}
\begin{tabular}{r r}
\bfseries Band & \bfseries Frequenz \\ \toprule
20 m & 14 100 MHz \\ \midrule                                                                                                                                                                 
17 m & 18 110 MHz \\ \midrule                                                                                                                                                                 
15 m & 21 150 MHz \\ \midrule                                                                                                                                                                 
12 m & 24 930 MHz \\ \midrule                                                                                                                                                                 
10 m & 28 200 MHz \\ \midrule
\end{tabular}
\vspace{1em}

Damit in alle Richtungen gleich viel Leistung abgestrahlt wird, werden vertikale Antennen ver­wendet. 

Die Baken senden in den ihnen zugeteilten 10 Sekunden zuerst ihr eigenes Ruf­zeichen und danach vier «Striche» von je einer Sekunde Dauer. Der erste wird mit 100 W gesendet, der zweite mit 10 W, der dritte mit einem Watt und der vierte mit 0.1 Watt. So kann man sich in etwa ein Bild davon machen, wie die aktuellen Ausbreitungsbedingungen sind.
Der Sendeplan der Baken sieht folgendermassen aus:


{
\newlength{\wRule}
\setlength{\wRule}{1pt}
\setlength{\belowrulesep}{0pt}
\setlength{\aboverulesep}{0pt}
\setlength{\cmidrulewidth}{\wRule}
\setlength{\lightrulewidth}{0pt}
\newcommand{\lrule}[1]{\multicolumn{1}{!{\vrule width \wRule} @{\hspace{4pt}} l @{}}{#1}}
\newcommand{\lrrule}[1]{\multicolumn{1}{!{\vrule width \wRule} @{\hspace{4pt}} l @{} !{\vrule width \wRule}}{#1}}
\newcommand{\rrule}[1]{\multicolumn{1}{@{\hspace{4pt}} l @{} !{\vrule width \wRule}}{#1}}
\newcommand{\vd}{\rotatebox{90}{\tiny{…}}}
\newcommand{\ta}{00, 03 \tiny{…}}
\newcommand{\tb}{01, 04 \tiny{…}}
\newcommand{\tc}{02, 05 \tiny{…}}

\definecolor{colA}{gray}{.1}
\definecolor{colB}{gray}{.2}
\definecolor{colC}{gray}{.3}
\definecolor{colD}{gray}{.4}
\definecolor{colE}{gray}{.5}
\newcommand{\pa}{\textcolor{colA}{+ 10 s}}
\newcommand{\pb}{\textcolor{colB}{+ 20 s}}
\newcommand{\pc}{\textcolor{colC}{+ 30 s}}
\newcommand{\pd}{\textcolor{colD}{+ 40 s}}
\newcommand{\pe}{\textcolor{colE}{+ 50 s}}

\noindent
\small{
\begin{tabular}{r @{\hspace{4pt}} l @{\hspace{4pt}} l @{\hspace{6pt}} l @{\hspace{6pt}} l @{\hspace{6pt}} l @{\hspace{6pt}} l @{\hspace{4pt}} r}
\bfseries RZ & \bfseries Land & \multicolumn{5}{c}{\textbf{Frequenz, Sendezeit} nach voller Stunde} & \bfseries km \\ [1.5ex]
\arrayrulecolor{white} \midrule \arrayrulecolor{black}

              &        & 14 100 kHz & 18 110 kHz & 21 150 kHz & 24 930 kHz & 28 200 kHz & zu Köln \\ [1.2ex]
                         \cmidrule{3-7}
UN            & 4u1un  & \lrule{\ta}        & \lrrule{\pa}        & \rrule{\pb}        & \rrule{\pc}        & \rrule{\pd}        & 6050 \\ [1ex]
Kanada        & ve8at  & \lrule{\pa}        & \lrrule{\pb}        & \rrule{\pc}        & \rrule{\pd}        & \rrule{\pe}        & 4550 \\ [1ex]
USA           & w6wx   & \lrule{\pb}        & \lrrule{\pc}        & \rrule{\pd}        & \rrule{\pe}        & \rrule{\tb}        & 9050 \\ [1ex]
Hawaii        & kh6wo  & \lrule{\pc}        & \lrrule{\pd}        & \rrule{\pe}        & \rrule{\tb}        & \rrule{\pa}        & 11 890 \\ [1ex]
Neuseeland    & zl6b   & \lrule{\pd}        & \lrrule{\pe}        & \rrule{\tb}        & \rrule{\pa}        & \rrule{\pb}        & 18 680 \\ [1ex]
Australien    & vk6rbp & \lrule{\pe}        & \lrrule{\tb}        & \rrule{\pa}        & \rrule{\pb}        & \rrule{\pc}        & 14 050 \\ [1ex]
Japan         & ja2igy & \lrule{\tb}        & \lrrule{\pa}        & \rrule{\pb}        & \rrule{\pc}        & \rrule{\pd}        & 9390 \\ [1ex]
Russland      & rr9o   & \lrule{\pa}        & \lrrule{\pb}        & \rrule{\pc}        & \rrule{\pd}        & \rrule{\pe}        & 4900 \\ [1ex]
Hong Kong     & vr2b   & \lrule{\pb}        & \lrrule{\pc}        & \rrule{\pd}        & \rrule{\pe}        & \rrule{\tc}        & 9300 \\ [1ex]
Sri Lanka     & 4s7b   & \lrule{\pc}        & \lrrule{\pd}        & \rrule{\pe}        & \rrule{\tc}        & \rrule{\pa}        & 8250 \\ [1ex]
Südafrika     & zs6dn  & \lrule{\pd}        & \lrrule{\pe}        & \rrule{\tc}        & \rrule{\pa}        & \rrule{\pb}        & 8780 \\ [1ex]
Kenia         & 5z4b   & \lrule{\pe}        & \lrrule{\tc}        & \rrule{\pa}        & \rrule{\pb}        & \rrule{\pc}        & 6780 \\ [1ex]
Israel        & 4x6tu  & \lrule{\tc}        & \lrrule{\pa}        & \rrule{\pb}        & \rrule{\pc}        & \rrule{\pd}        & 3100 \\ [1ex]
Finnland      & oh2b   & \lrule{\pa}        & \lrrule{\pb}        & \rrule{\pc}        & \rrule{\pd}        & \rrule{\pe}        & 1570 \\ [1ex]
Madeira       & cs3b   & \lrule{\pb}        & \lrrule{\pc}        & \rrule{\pd}        & \rrule{\pe}        & \rrule{\ta}        & 2750 \\ [1ex]                                         
Argentinien   & lu4aa  & \lrule{\pc}        & \lrrule{\pd}        & \rrule{\pe}        & \rrule{\ta}        & \rrule{\pa}        & 11 400 \\ [1ex]
Peru          & oa4b   & \lrule{\pd}        & \lrrule{\pe}        & \rrule{\ta}        & \rrule{\pa}        & \rrule{\pb}        & 10 600 \\ [1ex]
Venezuela     & yv5b   & \lrule{\pe}        & \lrrule{\ta}        & \rrule{\pa}        & \rrule{\pb}        & \rrule{\pc}        & 7950 \\ [1ex] 
                         \cmidrule{3-7}
\end{tabular}
}
}

\paragraph{Bake DK0WCY} sendet von Scheggerott aus (QTH JO44VQ) auf 10 144 MHz rund um die Uhr in CW, RTTY und PSK31 (siehe \href{http://www.dk0wcy.de/schedule_en.html}{dk0wcy.de}).

\paragraph{5-Mhz-Baken} in der Nähe:

\vspace{.5em}
\begin{tabular}{ll}
5195.0 kHz & DRA5 (Scheggerott) \\
5269.5 kHz & VO1MRC \\
5290.0 kHz & GB3RAL (nr Didcot) / GB3WES (Cumbria) / GB3ORK (Orkney)
\end{tabular}

\subsection{Ein-Buchstaben-Baken}
\lit{Dokument der Bandwacht: \href{http://www.iarums-r1.org/bandwacht/diverses/baken.pdf}{iarums-r1.org/bandwacht/diverses/baken.pdf}}
Auf einigen Frequenzen, unter anderem auf 7039 kHz, hat das Russische Militär Baken stationiert, die fortlaufend den selben Buchstaben im Morsecode senden. Über die Sendeleistung ist nichts bekannt, die Standorte hingegen konnten bestimmt werden.

Bekannte Frequenzen (nach dem Bandwacht-Dokument) sind in kHz: 3070, 3088, 3090, 3564, 3649, 3658, 3699, 4558, 5154, 7039, 8495, 10 872, 13 258, 13 528, 16 332, 17 016 und 20 048. 

%\newcommand\cyrtext[1]{{\fontencoding{T2A}\selectfont #1}}
%\newcommand{\ru}[1]{\foreignlanguage{russian}{#1}}

\begin{tabular}{rll}
\bfseries Buchstabe & \bfseries Kyrillisch & \bfseries Standort \\ \toprule
c & \ru{ц} & \ru{Москва} (Moskau) \\ \midrule
d & \ru{л} & \ru{Одеса} (Odessa; Ukraine) \\ \midrule
f & \ru{ф} & \ru{Владивосток} (Wladiwostok) \\ \midrule
k & \ru{к} & \ru{Хаьаровск} (Chabarowsk) \\ \midrule
l & \ru{л} & \ru{Санкт-Петербург} (Sankt Petersburg) \\ \midrule
m & \ru{м} & \ru{Магадан} (Magadan) \\ \midrule
o & \ru{о} & \ru{Москва} \\ \midrule
p & \ru{п} & \ru{Калининграл} (Kaliningrad) \\ \midrule
r & \ru{я} & \ru{Ижевск} (Ischewsk) \\ \midrule
s & \ru{с} & \ru{Архангелск} (Archangelsk) \\ \midrule
u & \ru{у} & \ru{Мурманск} (Murmansk) \\ \midrule
v & \ru{ж} & \ru{Xiva} (Chiwa; Usbekistan) \\ \midrule
ü & \ru{ю} & \ru{Холмск} (Kholmsk) \\ \midrule
z & \ru{з} & ? \\ \midrule
\end{tabular}

\subsection{Zeitzeichen-Frequenzen auf Kurzwelle}
Eine aktuelle Liste ist unter Anderem unter \href{http://de.wikipedia.org/wiki/Zeitzeichensender}{Wikipedia:Zeitzeichensender} und auf sfmeter.net\footnote{\href{http://www.smeter.net/stations/hf-time-frequency.php}{http://www.smeter.net/stations/hf-time-frequency.php}} zu finden.

Da die meisten Zeitzeichensender von einem Standort aus gleich auf mehreren Frequenzen senden, kann man sich so einen Überblick über die Ausbreitungsbedingungen schaffen. In Europa ist RWM gut erreichbar.

{
\newcommand{\desc}[2]{\begin{varwidth}[t]{6em} \hfill \textbf{#1}\\  \footnotesize \raggedleft #2 \end{varwidth}}
\newcommand{\freq}[2]{\parbox[t]{6em}{#1\\ \footnotesize #2}}
\newcommand{\mul}[1]{\begin{varwidth}[t]{6em} #1 \end{varwidth} }

\begin{longtabu} to \linewidth{rl r p{6cm}}
\bfseries Call & \desc{Frequenzen}{kHz} & \desc{Power}{[kW]} & \bfseries Beschreibung \\ \toprule
\endhead
\arrayrulecolor{rowsep}
WWV & \mul{2500\\ 5000\\ 10\,000\\ 15\,000\\ 20\,000} & \mul{2.5\\ 10\\ 10\\ 10\\ 2.5} & Colorado, USA, vom NIST (National Institute of Standards and Technology). Zeitinformationen werden auch in Voice ausgegeben.\footnote{Siehe \href{http://www.nist.gov/pml/div688/grp40/wwv.cfm}{http://www.nist.gov/pml/div688/grp40/wwv.cfm}} \\ \midrule
WWVH & \mul{2500\\ 5000\\ 10\,000\\ 15\,000} & \mul{5\\ 10\\ 10\\ 10} & Hawaii, USA. (NIST; wie oben.)\footnote{Siehe \href{http://tf.nist.gov/timefreq/stations/wwvh.htm}{http://tf.nist.gov/timefreq/stations/wwvh.htm}}\\ \midrule
CHU & \mul{3330\\ 7850\\ 14\,670} & \mul{3\\ 10\\ 3} & Ottawa, Kanada. Bis Ende 2008 wurde auf 7335\,kHz statt 7850\,kHz gesendet.\footnote{Siehe \href{http://www.nrc-cnrc.gc.ca/eng/services/time/short\_wave.html}{http://www.nrc-cnrc.gc.ca/eng/services/time/short\_wave.html} }\\ \midrule
RWM & \mul{4996\\ 9996\\ 14996} & \mul{5\\ 8\\ 8} & Moskau, Russland. Ein Sendeplan befindet sich auf der Englischen \href{http://en.wikipedia.org/wiki/RWM}{Wikipedia:RWM}.\\ \midrule
YVTO & \mul{5000} & \mul{1} & Venezuela. Nur Voice. \\ \midrule


\end{longtabu}

}

\section{Verfahrenspezifische Frequenzen}
\subsection{QRP-Frequenzen}
Auf diesen Frequenzen wird mit verringerter Leistung (≤ 5 W) gesendet.

\vspace{1em}
\noindent
\begin{tabular}{r r}
\bfseries CW & \bfseries SSB \\ \toprule \arrayrulecolor{rowsep}
 1 810 kHz &  1 843 kHz \\ \midrule
 3 560 kHz &  3 690 kHz \\ \midrule
 7 030 kHz &  7 060 kHz \\ \midrule
10 106 kHz &          — \\ \midrule
14 060 kHz & 14 285 kHz \\ \midrule
18 106 kHz & 18 130 kHz \\ \midrule
21 060 kHz & 21 285 kHz \\ \midrule
24 906 kHz & 24 950 kHz \\ \midrule
28 060 kHz & 28 360 kHz \\ \midrule
\end{tabular}

\subsection{QRS-Frequenzen CW}
Auf diesen Frequenzen (plus 5 kHz aufwärts) wird etwas langsamer gemorst. Sie sind speziell für Einsteiger sind sie empfehlenswert.

\vspace{1em}
\noindent
\begin{tabular}{r}
  \textbf{QRS} \\ \toprule \arrayrulecolor{rowsep}
  3555 kHz \\ \midrule
  7055 kHz \\ \midrule
  7155 kHz \\ \midrule
14 055 kHz \\ \midrule
21 055 kHz \\ \midrule
28 055 kHz \\ \midrule
\end{tabular}

\subsection{Frequenzen für PSK}
Die folgenden Frequenzen werden häufig für PSK verwendet und sind daher eine gute Wahl als Aufruffrequenz.

\vspace{1em}
\noindent
\begin{tabular}{rl@{\hspace{2em}}r}
\bfseries Frequenz & \bfseries Region & \bfseries Frequenz \\ \toprule \arrayrulecolor{rowsep}
1838 kHz &  & 14 070 kHz \\ \cmidrule{1-1} \cmidrule{3-3}
3580 kHz &  & 18 100 kHz \\ \cmidrule{1-1} \cmidrule{3-3}
7035 kHz & Region 1/3 & 21 080 kHz \\ \cmidrule{1-1} \cmidrule{3-3}
7071 kHz & USA & 24 920 kHz \\ \cmidrule{1-1} \cmidrule{3-3}
7080 kHz & Region 2 & 28 070 kHz \\ \cmidrule{1-1} \cmidrule{3-3}
10 142 kHz &  & 28 120 kHz \\ \cmidrule{1-1} \cmidrule{3-3}
\end{tabular}

\subsection{SSTV- und FAX-Frequenzen}
\noindent
\begin{tabular}{r}
\bfseries Frequenz \\ \toprule \arrayrulecolor{rowsep}
  3730 kHz \\ \midrule
  7040 kHz \\ \midrule
14 230 kHz \\ \midrule
21 340 kHz \\ \midrule
28 680 kHz \\ \midrule
\end{tabular}

\subsection{Pactor-Mailboxen}
\paragraph{HB9AK} bei JN47FE scannt die Frequenzen 7051.5, 10145.9, 14110.4 und 18115.9 kHz.
\paragraph{HB9HAI} bei JN46VU scannt alle Frequenzen von 3.5 bis 28 MHz.
\paragraph{HB9IAC} bei JN36DF scannt 3582--3586, 7038--7040, 10\,142--10\,146, 14\,068--14\,076, 21\,072--21\,083 und 28\,087--28\,103\,kHz.
\paragraph{HB9XQ} bei JN36PV scannt 3613, 7054 und 14\,119 kHz.
\paragraph{HB9MM} bei JN36EH scannt 3605.5, 7053.0, 10\,145.0 und 14\,109.2\,kHz.


\section{Spezielle Aktivitäten}

\subsection{Schweizer Amateurfunkativitäten}
Siehe auch: \href{http://funkperlen.wordpress.com/2010/11/22/schweizer-runden/}{http://funkperlen.wordpress.com/2010/11/22/schweizer-runden/}

\begin{longtabu} to \linewidth{r p{2cm} r p{5cm}}
\rowfont \bfseries HBT\footnote{Abkürzung für HB Time, also Lokalzeit in der Schweiz} & Wochentag & Frequenz & Region, Mode \\ \toprule
\endhead \arrayrulecolor{rowsep}
20:15 Uhr & Letzter Sonntag im Monat & 145.6625 MHz & Rheintal \\ \midrule
10:30 Uhr & Sonntag   & 51.490 MHz & Schweiz; FM \\ \midrule
10:45 Uhr & Sonntag   & 50.163 MHz & Schweiz; USB \\ \midrule
24 h      & Täglich   & 14.347 MHz & RARES (Russian Amateur Radio Emer­gency Service), Notruffrequenz \\ \midrule
08:00 Uhr & Werktags  & 14.335 MHz & Australienrunde \\ \midrule
24 h      & Täglich   & 14.292 MHz & RARES-Meetings; SSB/CW \\ \midrule
12:30 Uhr & Werktags   & 14.317 MHz & Auslandschweizer \\ \midrule
23:00 Uhr & Sommer    & 14.172 MHz & Seeräuberrunde \\ \midrule
22:00 Uhr & Winter    & 14.172 MHz & Seeräuberrunde \\ \midrule
08:45 Uhr & Jeden 2. Sonntag & 3775 MHz & USKA News \\ \midrule
09:00 Uhr & Sonntag   & 3775 MHz & Schweizer-Runde \\ \midrule
10:00 Uhr & Sonntag   & 3710 MHz & Flughafenrunde FHR \\ \midrule
18:15 Uhr & Täglich   & 3747.5 MHz & Bernerrunde \\ \midrule
20:30 Uhr & Sonntag   & 3695 MHz & Rheintal \\ \midrule
24 h      & Täglich   & 3637 MHz & Chaotentrunde \\ \midrule
19:00 Uhr & Montag    & 3574 kHz & Morseübungssendung vom HTC,  40–140 BpM (8–28 wpm) \\ \midrule
20:05 Uhr & Montag    & 1991 MHz & Montagsrunde, Zürich \\ \midrule
\end{longtabu}

\subsection{SOTA-Frequenzen}
Berg-Aktivitäten (Funk von Berg zu Berg) finden auf folgenden SOTA-Vorzugsfrequenzen statt:

\vspace{1em}
\begin{tabular}{lrrrrrrrrl}
 \bfseries CW & 28 060 & 24 906 & 21 060 & 18 096 & 14 060 & 10 106 & 7030 & 3560 & kHz \\ \toprule
\bfseries SSB & 28 365 & 24 950 & 21 285 & 18 130 & 14 285 & — & 7090 & 3690 & kHz
\end{tabular}


\subsection{IOTA-Frequenzen}
Insel-Aktivitäten finden auf folgenden IOTA-Vorzugsfrequenzen statt:

\vspace{1em}
\begin{tabular}{lrrrrrrrrl}
\bfseries CW & 28 040 & 24 920 & 21 040 & 18 098 & 14 040 & 10 115 & 7030 & 3530 & kHz \\ \toprule
\bfseries SSB & 28 560 & 28 460 & 24 950 & 21 260 & 18 128 & 14 260 & 7055 & 3755 & kHz
\end{tabular}
\vspace{1em}

Diese Frequenzen müssen für Verbindungen zwischen Inseln freigehalten werden.

\subsection{Flugfunk}
Auf VHF/UHF werden für Flugfunk (AM!) folgende Frequenzen verwendet:

\vspace{1em}
\begin{tabular}{rcll}
\multicolumn{3}{c}{\textbf{Frequenz}} & \textbf{Verwendung} \\ \toprule \arrayrulecolor{rowsep}
118 & — & 137 MHz & zivil \\ \midrule
138 & — & 144 MHz & mil \\ \midrule
241.5 & — & 386 MHz & mil/ziv \\ \midrule
\end{tabular}
\vspace{1em}

Flugfunk darf in der Schweiz mitgehört werden.

\section{Funkdienste}
\subsection{Radio-Frequenzen auf Kurzwelle}
Folgende Rundfunkbänder sind für Radiosender auf Kurzwelle vorgesehen:

\vspace{1em}
\begin{tabular}{>{\bfseries} r r@{}c@{}l @{\hspace{4pt}}l}
\bfseries Band & \multicolumn{3}{c}{\bfseries Frequenz} \\ \toprule \arrayrulecolor{rowsep}
49 m & 5950 & — & 6200 & kHz \\ \midrule
41 m & 7100 & — & 7300 & kHz \\ \midrule
31 m & 9500 & — & 9900 & kHz \\ \midrule
25 m & 11 650 & — & 12 050 & kHz \\ \midrule
19 m & 13 600 & — & 13 800 & kHz \\ \midrule
16 m & 15 100 & — & 15 600 & kHz \\ \midrule
13 m & 21 450 & — & 21 850 & kHz \\ \midrule
11 m & 25 670 & — & 26 100 & kHz  \\ \midrule
\end{tabular}
\vspace{1em}

Die mesten Stationen senden jeweils nur zu bestimmten Tageszeiten. Ein Sendeplan ist zum Beispiel auf \href{http://www.fading.de}{www.fading.de} zu finden.

\subsection{Deutscher Wetterdienst}
Der Deutsche Wetterdienst sendet Informationen zum Seewetter auf 4583, 7646 und 10 100.8\,kHz (RTTY, F1B, 50 Bd, 450 Hz Shift, invertiert) und Wetterkarten auf 3855, 7880 und 13882.5 kHz (Faksimile, F1C). Über RTTY werden abwechslungsweise codierte Wetterdaten für Computerprogramme und Klartext gesendet.

\subsection{Volmet-Frequenzen auf Kurzwelle}
Auf VOLMET-Frequenzen werden meteorologische Informationen, die für Start/Landung auf Flugplätzen benötigt werden, übertragen. 

\vspace{1em}
\begin{tabular}{lrclll}
\bfseries Gebiet & \bfseries Frequenz & \bfseries UTC & \bfseries Station & \bfseries Call & \bfseries Land \\ \toprule \arrayrulecolor{rowsep}
Africa & 10 057 kHz & 25, 55 & Brazzville & TNL & CGO \\ \midrule
 &  &  &  &  &  \\ \midrule
Europa & 3413 kHz & 00, 30 & Shannon & EIP & IRL \\ \midrule
 & 5505 kHz & 00, 30 & Shannon & EIP & IRL \\ \midrule
 & 8957 kHz & 00, 30 & Shannon & EIP & IRL \\ \midrule
 & 1326 kHz & 00, 30 & Shannon & EIP & IRL \\ \midrule
 &  &  &  &  &  \\ \midrule
North Atlantic & 3485 kHz & 00, 30 & New York & WSY70 & USA \\ \midrule
 & 6604 kHz & 20, 50 & Gander & VFG & CAN \\ \midrule
 & 10 051 kHz &  &  &  &  \\ \midrule
 & 13 270 kHz &  &  &  &  \\ \midrule
\end{tabular}










