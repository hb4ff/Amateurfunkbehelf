\chapter{Vorwort}
Das Ziel dieses Behelfs ist es, wichtige Informationen in kompakter Form bereitzustellen, speziell für mobilen oder portablen Einsatz\footnote{Am praktischsten ist er im Format A5.} (Ermöglicht wird dies auch durch die bereitgestellten Anhänge, wofür wir uns bedanken). Er ist aus den Grund entstanden, dass schlicht und einfach die Notwendigkeit daran bestand, aber noch nichts dergleichen existierte.

\section{Lizenz}
Frei zugängliches Wissen ist die Zukunft, wie auch die Wikipedia beweist. Aus diesem Grund steht dieser Behelf (inklusive Grafiken) und unter der GFDL, der GNU Free Documentation License. Das heisst unter anderem, dass er auch abgeändert und weiterverwendet werden darf (etwa für andere deutschsprachige Länder), unter der Bedingung, dass er weiterhin unter dieser Lizenz stehen muss. Genaueres unter \plink{sec:gfdl}{GNU Free Documentation License}.

\section{Prüfung}
\lit{Amateurfunk-Lehrgang von Eckart Moltrecht. Drei Bücher, auch online verfügbar: \href{http://www.dj4uf.de}{dj4uf.de}}

Dieser Behelf ist primär als Nachschlagewerk und nicht zur Vorbereitung für die HB3- oder HB9-Prüfung gedacht! Dafür ist spezielle Literatur besser geeignet. Die USKA bietet zum Beispiel eine CD mit Prüfungsfragen an.

\section{Typografie}
\lit{PDF-Dokument über Typografie: «typokurz -- Einige wichtige typografische Regeln» von Christoph Bier, \href{http://zvisionwelt.wordpress.com/downloads/}{zvisionwelt.wordpress.com/downloads/}}

Damit Informationen schneller gefunden werden können, wurden bestimmte Textteile hervorgehoben. Stichwörter, die im Glossar erklärt werden, werden rot und mit der Seitenzahl direkt nach dem Doppelpunkt geschrieben (Beispiel: \tlink{term:rit}{RIT}). Für Begriffe, die im Text näher erklärt werden, wird die ausführliche Schreibeweise verwendet (Beispiel: \plink{sec:antennen}{Antennen}). Interessante weiterführende Literatur wird mit dem Buchsymbol (siehe oben) markiert. Beispiele (etwa bei Rufzeichen) werden \cw{grün und in Monospace} geschrieben.

Ausserdem wurde für eine bessere Lesbarkeit Wert auf korrekte Typografie gelegt (siehe Literaturhinweis), also etwa zwischen Bindestrichen, Gedankenstrichen und Minuszeichen unterschieden und «Schweizer Anführungszeichen» verwendet. 

\section{Ergänzungen und Korrekturen}
\dots~sind herzlich willkommen. Der Behelf kann auch direkt auf GitHub\footnote{\href{https://github.com/hb4ff/Amateurfunkbehelf}{https://github.com/hb4ff/Amateurfunkbehelf}} verändert werden. Mails werden unter simon.eu@gmail.com entgegengenommen. Ansprechperson: Simon Eugster. 

\section{Letzte Änderungen}
Kurze Liste der Änderungen seit den letzten Versionen:
\begin{itemize}
 \item[Feb 13] Umstellung auf LaTeX; Ergänzungen zu Koaxkabel, Bandleitung, Notruf, QSL-Karten, Karten RU/UK
 \item[v8.1]   Links klickbar im PDF, Korrekturen und Ergänzungen, Bandplan aktualisiert, Raumwelle ergänzt, neu: Übersicht über verschiedene Frequenzbereiche
 \item[v8]     Glossar/Tone Call, Glossar/Echolink – 8.1 Pactor-Frequenzen, Ein-Buchstaben-Baken, Zeitzeichen auf HF
\end{itemize}




