{
\usefont{T1}{DejaVuSans-TLF}{m}{n}

\cleardoublepage
\vspace*{3cm}
\noindent
Copyright (c) \hspace{1em} 2008--2013 \hspace{1em} hb4ff Operators and Authors.

\vspace{1em}
\noindent
Permission is granted to copy, distribute and/or modify this document\\
under the terms of the \textbf{GNU Free Documentation License,} Version 1.3\\
or any later version published by the Free Software Foundation;\\
with no Invariant Sections, no Front-Cover Texts, and no Back-Cover Texts.\\
A copy of the license is included in the section entitled «GNU\\
Free Documentation License».

\vspace{2em}
\noindent
Dieser Behelf sowie die Karten, Grafiken und Vorlagen \\
können unter \textit{\href{http://ham.granjow.net}{http://ham.granjow.net}} heruntergeladen werden.

\vspace{2em}
\noindent
Die Autoren übernehmen keine Haftung \\
für in Zusammenhang mit diesem Dokument entstandene Schäden.\\
Der Leser handelt in eigener Verantwortung.\\

\cleardoublepage
\vspace*{3cm}
\begin{centering}
Dieser Behelf wurde von Amateurfunkern der Station HB4FF\\
ins Leben gerufen.

\vspace{1em}
Diese Ausgabe ist vom \today.

\vspace{2em}
\textbf{Autoren}\\ \vspace{4pt}
Jannick Griner\\
Jürg Rüfli, hb9bfc\\
Markus Walter, hb9hvg\\
Simon A. Eugster, hb9eia

\newcommand{\dank}[2]{\vspace{4pt}#1\\{\footnotesize #2} \\}
\vspace{2em}
\textbf{Dank}\\
\dank{Anja Ballschmieter}{für das Logo auf der Titelseite}
\dank{Stephan Bolli, hb9tnp}{für die HB9-Formelsammlung}
\dank{Paul-Adrien Braissant, hb9cuf}{für die Hinweise und Ergänzungen zu Modulation und Verfahren}
\dank{Josef Buchs, hb9mty}{für die ergänzenden Beschreibungen der Modulationsverfahren}
\dank{Karl Haab, hb9aiy}{für die technischen Ergänzungen}
\dank{Peter Kumli, Bakom}{für die Bereitstellung der Vorschriften}
\dank{Oona Svan}{für die Grafik mit dem Maidenhead-Locator}
\dank{Renato Schlittler, hb9bxq}{für die Relaislisten}
\dank{Gregor Wuthier, hb9dmh}{für die Anregungen zum Inhalt}
\end{centering}

}