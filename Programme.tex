\chapter{Programme}

\section{Linux}
\paragraph{Baudline} Sehr umfangreiche Spektrumanalyse mit vielen Anzeigen (wie Spektrum, Wasserfall, Durchschnitt) und Einstellungsmöglichkeiten wie ein Rauschfilter. Beispiel: \link{Grafik 12, RTTY-Wasserfall}.

\paragraph{extcalc} Rechner, der auch zur grafischen Darstellung von Modulationen verwendet werden kann. Unterstützt Integral etc. Beispiel: \link{Grafik 4 zur Amplitudenmodulation}.

\paragraph{FFTExplorer} Spektrumanalyse mit der Möglichkeit, gewisse Signale zu generieren und zu modulieren (AM/FM). \href{http://www.arachnoid.com}{www.arachnoid.com}

Auf der Seite sind ausserdem weitere interessante Programme zu finden wie eines zum lokalisieren von Satelliten oder zum generieren von Tönen.

\paragraph{japa} (JACK and ALSA Perceptual Analyser) Spektrumanalyse (z.\,B. \link{Grafik 5}, AM-Signal mit dem typischen Peak des Trägersignals). \href{http://www.kokkinizita.net/linuxaudio/}{www.kokkinizita.net/linuxaudio/}

\paragraph{jaaa} (JACK and ALSA Audio Analyser) Ähnlich wie japa, selber Link.

\paragraph{Audacity} Audio-Editor. Auch für Mac und Windows verfügbar. Beispiel: \link{Grafik 13}, Ausschnitt eines PSK31-Datenstroms. \href{http://www.audacity.de}{www.audacity.de}

\section{Windows}
\paragraph{SpecLab 20001} Wasserfall- und andere Diagramme, Analysemöglichkeiten

\paragraph{4nec212} Antennensimulationsprogramm. \href{http://home.ict.nl/$\sim$arivoors/}{http://home.ict.nl/$\sim$arivoors/}

\paragraph{Great Circular Maps12} Projiziert ein Maidenhead-Locator-Netz auf die Erdkugel. \href{http://hem.passagen.se/sm3gsj/gcm/download.htm}{http://hem.passagen.se/sm3gsj/gcm/download.htm}