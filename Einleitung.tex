\chapter{Was ist Amateurfunk?}
Hierzu ein kurzes Zitat:
\begin{quotation}
 Stell Dir vor, wie gering das Licht einer Taschenlampe ist. Die Leistung von ca. 3 Watt reicht knapp, um damit nachts einige Meter weit zu leuchten. Kannst Du Dir auch vorstellen, dass man mit diesen 3 Watt Leistung eine Funkverbindung über Hunderte oder sogar Tausende von Kilometer herstellen kann? Man hat dabei den Eindruck, die Physik überlistet zu haben, denn das funktioniert tatsächlich, Faszination pur.
 
 \hspace{\fill} --- \textit{http://www.uska.ch, 2008}
\end{quotation}


Amateurfunk – dazu gehören Elektronik, Sport, Computer, Kultur, Katastrophenhilfe (Stichwort New Orleans), weltweite Verbindungen, \dots\ (die Liste könnte man beinahe beliebig weiterführen).

Amateurfunk ist ein Hobby, das auf der ganzen Welt anzutreffen ist – in den einen Ländern häufiger, in anderen wie China aufgrund politischer Umstände eher selten. Es ist ein sehr vielseitiges Hobby. An erster Linie steht natürlich die Technik, die die drahtlose Übertragung erst ermöglicht. Amateure erlangen daher schnell viel Wissen auf diesem Gebiet.

Ums Experimentieren kommt man kaum herum. Nur schon die sich ständig wechselnden Ausbreitungsbedingungen verändern die Reichweite, die man auf bestimmten Frequenzbändern erzielen kann, auf HF von wenigen hundert bis zu mehreren tausend Kilometern.

Regelmässig finden auf verschiedenen Bändern Contests statt, bei denen gilt, möglichst viele Verbindungen herzustellen. Je nach Art des Contests werden die Bedingungen auch erschwert, indem etwa nur tragbare Ausrüstung erlaubt ist.

Eine weitere Aktivität ist die sogenannte Fuchsjagd. Hier werden in einem Gelände Sender verteilt, die wie bei einem OL möglichst schnell gefunden werden müssen. Erschwerend kommt jedoch hinzu, dass die Positionen nicht auf der Karte vermerkt sind, sondern selber durch Peilungen bestimmt werden müssen.

Ein wichtiger Punkt ist der sogenannte «Ham Spirit».

Amateurfunker sind übrigens die Einzigen, die Sender selber bauen und ohne externe Prüfung in Betrieb nehmen dürfen!

\section{Der «Ham Spirit»}
Im Jahre 1928 schrieb Paul M.~Segal, w9eea, den «Amateur's Code», der Amateurfunker beschreibt:

\begin{quotation}
\textit{The Amateur's Code}

\vspace{1em}
\noindent
\textit{The Radio Amateur is:}\\
\textit{Considerate …} never knowingly operates in such a way as to lessen the pleasure of others. \\
\textit{Loyal …} offers loyalty, encouragement and support to other amateurs, local clubs, and his or her national radio amateur association. \\
\textit{Progressive …} with knowkedge abreast of science, a well-built and efficient station and operation above reproach. \\
\textit{Friendly …} slow and patient operating when requested; friendly advice and counsel to the beginner; kindly assistance, cooperation and \\ consideration for the interest of others. These are the hallmarks of the amateur spirit. \\
\textit{Balanced …} radio is an avocation, never interfering with duties owed to family, job, school, or community. \\
\textit{Patriotic …} station and skill always ready for service to country and community.
 
\end{quotation}


\section{Rufzeichen}
Jeder lizenzierte Amateurfunker hat ein weltweit einzigartiges Rufzeichen (Siehe \link{Rufzeichen}, Seite 44), genau wie Flugzeuge oder Schiffe. Schweizer Rufzeichen beginnen normalerweise mit hb3 (geringere Anzahl Frequenzbänder) oder (CEPT-Ausweis) hb9.

\section{Die QSL-Karten}
Bei direkten Verbindungen zwischen Stationen (d.\,h. nicht über ein Relais oder übers Internet) werden sogenannte QSL-Karten ausgetauscht. QSL ist einer von vielen Q-Codes (siehe \link{Die im Amateurfunk gebräuchlichsten Q-Codes}, Seite 16), welche vor Allem in CW der Abkürzung des Funkverkehrs dienen; er bedeutet «Empfangsbestätigung». Diese Karten sind von jeder Station einzigartig und werden daher auch gesammelt. Karten aus Ländern mit wenigen Amateurfunkern sind natürlich entsprechend begehrt, aber auch solche von «DX-peditions», Expeditionen in entfernte Gegenden wie beispielsweise der Antarktis. 

Auf der Vorderseite ist meist das Rufzeichen und dazu eine Grafik oder ein Foto der eigenen Station oder der Umgebung abgedruckt. Die Rückseite wird vor dem Versand ausgefüllt: Call des Empfängers, Datum und Zeit in UTC, Frequenzband, Übertragungsverfahren, Rapport (RST), Ort, Gruss und Unterschrift. Der Ort kann auch mit einem \link{Maidenhead-Locator} angegeben werden.

\section{UTC?}
Würde jeder beim Funken die Ortszeit verwenden (Logbuch, QSL-Karte, …), gäbe dies ein riesiges Durcheinander. Daher werden alle Zeiten in der Standard-Zeit UTC (Coordinated Universal Time) angegeben. Die UTC findet auch zum Beispiel auf der Internationalen Raumstation ISS, in der Antarktis und in der Luft- und Seefahrt.

In der Schweiz ist die Ortszeit während der Winterzeit UTC+1, während des Sommers UTC+2. Im Sommer entspricht 18:00 Ortszeit (also 18:00 UTC+2) folglich 16:00 UTC.

Unsere Ortszeit wird auch als \textit{MEZ} (Mitteleuropäische Zeit) oder \textit{MESZ} (Mitteleuropäische Sommer­zeit) bezeichnet, bzw. auf Englisch \textit{CET} und \textit{CEST}.